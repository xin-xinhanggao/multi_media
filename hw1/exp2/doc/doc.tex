\documentclass{ctexart}
\usepackage{graphicx}
\usepackage{amsmath}
\usepackage{listings}
\usepackage{color}
\usepackage{makecell}
\usepackage{subfigure}

\definecolor{mygreen}{rgb}{0,0.6,0}
\definecolor{mygray}{rgb}{0.5,0.5,0.5}
\definecolor{mymauve}{rgb}{0.58,0,0.82}

\lstset{ %
  backgroundcolor=\color{white},   % choose the background color; you must add \usepackage{color} or \usepackage{xcolor}; should come as last argument
  basicstyle=\footnotesize,        % the size of the fonts that are used for the code
  breakatwhitespace=false,         % sets if automatic breaks should only happen at whitespace
  breaklines=true,                 % sets automatic line breaking
  captionpos=b,                    % sets the caption-position to bottom
  commentstyle=\color{mygreen},    % comment style
  deletekeywords={...},            % if you want to delete keywords from the given language
  escapeinside={\%*}{*)},          % if you want to add LaTeX within your code
  extendedchars=true,              % lets you use non-ASCII characters; for 8-bits encodings only, does not work with UTF-8
  frame=single,	                   % adds a frame around the code
  keepspaces=true,                 % keeps spaces in text, useful for keeping indentation of code (possibly needs columns=flexible)
  keywordstyle=\color{blue},       % keyword style
  language=Octave,                 % the language of the code
  morekeywords={*,...},           % if you want to add more keywords to the set
  numbers=left,                    % where to put the line-numbers; possible values are (none, left, right)
  numbersep=5pt,                   % how far the line-numbers are from the code
  numberstyle=\tiny\color{mygray}, % the style that is used for the line-numbers
  rulecolor=\color{black},         % if not set, the frame-color may be changed on line-breaks within not-black text (e.g. comments (green here))
  showspaces=false,                % show spaces everywhere adding particular underscores; it overrides 'showstringspaces'
  showstringspaces=false,          % underline spaces within strings only
  showtabs=false,                  % show tabs within strings adding particular underscores
  stepnumber=2,                    % the step between two line-numbers. If it's 1, each line will be numbered
  stringstyle=\color{mymauve},     % string literal style
  tabsize=2,	                   % sets default tabsize to 2 spaces
  title=\lstname                   % show the filename of files included with \lstinputlisting; also try caption instead of title
}

\author{2014011355 辛杭高}
\title{图像视频大作业一 Part II}
\begin{document}
\maketitle
本部分的主要工作是对lena.bmp进行DCT变换,并对变化后的DCT矩阵进行量化,考虑不同量化矩阵对量化效果的影响。


\section{Usage}
本部分实验全部由matlab代码编写完成,直接运行src文件夹下的exp2.m文件即可,运行时必须保证lena.bmp也在
src文件夹下,运行完毕后会输出一系列参数,并保存一系列图片。

会输出6个参数,这6个参数为分别为使用Q矩阵,Cannon矩阵和Nikon矩阵对DCT结果进行量化后的总体图片PSNR度量
值以及各块平均度量值。并且会保存三张图片Qrecover.png,Cannonrecover.png和Nikonrecover.png,这三张
图片为Q矩阵,Cannon矩阵和Nikon矩阵量化后的参数恢复后的图片。

除此之外,还会plot一张图,该图显示了恢复后图片的总体PSNR度量值随参数a的变化结果。

\section{Q矩阵量化}
量化用的Q矩阵为
\begin{equation}
\begin{bmatrix}
 16 & 11 & 10 & 16 & 24 & 40 & 51 & 61 \\
 12 & 12 & 14 & 19 & 26 & 58 & 60 & 55 \\
 14 & 13 & 16 & 24 & 40 & 57 & 69 & 56 \\
 14 & 17 & 22 & 29 & 51 & 87 & 80 & 62 \\
 18 & 22 & 37 & 56 & 68 & 109 & 103 & 77 \\
 24 & 35 & 55 & 64 & 81 & 104 & 113 & 92 \\
 49 & 64 & 78 & 87 & 103 & 121 & 120 & 101 \\
 72 & 92 & 95 & 98 & 112 & 100 & 103 & 99
\end{bmatrix}
\end{equation}
若我们预备进行处理的DCT参数矩阵为
\begin{equation}
\begin{bmatrix}
 -415 & -33 & -58 &  35 &  58 & -51 & -15 & -12 \\
    5 & -34 &  49 &  18 &  27 &   1 &  -5 &   3 \\
  -46 &  14 &  80 & -35 & -50 &  19 &   7 & -18 \\
  -53 &  21 &  34 & -20 &   2 &  34 &  36 &  12 \\
    9 &  -2 &   9 &  -5 & -32 & -15 &  45 &  37 \\
   -8 &  15 & -16 &   7 &  -8 &  11 &   4 &   7 \\
   19 & -28 &  -2 & -26 &  -2 &   7 & -44 & -21 \\
   18 &  25 & -12 & -44 &  35 &  48 & -37 & -3
\end{bmatrix}
\end{equation}
则对DCT矩阵进行量化处理后可得
\begin{equation}
\begin{bmatrix}
 -26 & -3 & -6 &  2 &  2 & -1 & 0 & 0 \\
   0 & -3 & 4 &  1 &  1 &  0 & 0 & 0 \\
  -3 &  1 &  5 & -1 & -1 &  0 & 0 & 0 \\
  -4 &  1 &  2 & -1 &  0 &  0 & 0 & 0 \\
   1 &  0 &  0 &  0 &  0 &  0 & 0 & 0 \\
   0 &  0 &  0 &  0 &  0 &  0 & 0 & 0 \\
   0 &  0 &  0 &  0 &  0 &  0 & 0 & 0 \\
   0 &  0 &  0 &  0 &  0 &  0 & 0 & 0
\end{bmatrix}
\end{equation}
从上述量化过程可知,Q矩阵主要会保留DCT矩阵的左上角,对于DCT矩阵的其余区域则有清零操作。Q矩阵进行量化操作的
合理性在于,DCT矩阵的左上角对应于图象的低频区域,由于人眼对于图象的低频区域更为敏感,对于高频区域不够敏感,
因此经过这种量化后的DCT矩阵依然能够还原出一张质量较高的图片。

\begin{figure}
\centering
\includegraphics[width=0.5\textwidth]{Qrecover.png}
\caption{经Q矩阵量化后DCT恢复所得图.}
\end{figure}

利用Q矩阵量化DCT参数后进行还原,可得上述图象,经过与原图的对比,可以测量得恢复后的图象PSNR度量值为
\begin{equation}
PSNR_Q = 36.0996
\end{equation}

若对图象每一个划分小块的PSNR值进行测量并对各结果进行平均,得到的结果为
\begin{equation}
PSNR_{Qaverage} = 74.4811
\end{equation}

采取Q矩阵进行量化,高频部分不会完全消除,一些重要的高频信号会被保留,相对于实验part I中直接丢弃高频分量
,Q矩阵量化从理论上和PSNR上都更优一些。

\section{参数a对矩阵量化的影响}
使用矩阵aQ对DCT结果进行量化,其中a的取值从0.1开始以0.1为单位递增直到2.0,对于DCT矩阵量化后的结果进行idct2恢复,计算恢复后图象相对于原图的PSNR值,可以得到PSNR量化图表如图2所示。

\begin{figure}
\centering
\includegraphics[width=0.5\textwidth]{Q-a.png}
\caption{Q矩阵量化后恢复结果的PSNR图表.}
\end{figure}
从图2中可见,随着a的增加,PSNR会逐步下降。当a很小的时候,DCT量化的结果不一定能有滤去高频分量的作用,
如此结果不能起到压缩的作用。当a很大的时候,不仅仅会滤去DCT的高频分量,还会滤去DCT的低频分量,从而
导致PSNR下降,但是压缩比会大大增加。因此,在选取参数a的时候,需要考虑压缩比和PSNR的共同影响。

另外,可以看出从0.2到0.4的信噪比会有显著下降,可以认为0.2足以能够保证画面的质量,0.4可以作为图象压缩
的参数,从[0.2,0.4]的参数可以用于调整拍摄后照片的质量和空间。

除了Q矩阵之外,还应考虑Cannon矩阵和Nikon矩阵随参数a的变化,如图3,图4所示。
\begin{figure}
\centering
\includegraphics[width=0.5\textwidth]{Cannon-a.png}
\caption{Cannon矩阵量化后恢复结果的PSNR图表.}
\end{figure}

\begin{figure}
\centering
\includegraphics[width=0.5\textwidth]{Nikon-a.png}
\caption{Nikon矩阵量化后恢复结果的PSNR图表.}
\end{figure}

从图3和图4可以看出,不同于Q矩阵,Nikon矩阵和Cannon矩阵的图表随着a的变化会存在拐点,并不会如同Q矩阵一样随着a的增加直线下降,出现这个现象一方面是因为Nikon矩阵和Cannon矩阵在数值上都比较小,另一方面是Nikon矩阵和Cannon矩阵在高频和低频分量上的数值差异更大。因此当a比较小的时候,Nikon矩阵和Cannon矩阵不足以磨去高频分量,只能
削弱高频分量,此时PSNR值较低,当a稍微大一些时,Nikon矩阵和Cannon矩阵可以磨去高频分量但不会过度损伤低频分量
,因此PSNR反而会上手,当a更大的时候,高频和低频分量会一起被磨去,所以PSNR会再次下降。

另外,从PSNR的角度来说,Nikon矩阵优于Cannon矩阵优于Q矩阵。

从照相机的角度来说Nikon矩阵和Cannon矩阵都有两个极小值,在不能将a值调低获得高PSNR的情况下,可以使a值在两个
极小值之间调节,从而获得不错的图片质量(从PSNR角度来说),也能获得不错的压缩比。
\section{设计量化矩阵需要考虑的因素}
需要考虑待处理图像的类型,不同类型的图像高频与低频的成分不同,因此设计量化矩阵时应该注意待处理图像的类型。比如天文类的图像和柔和的自然景观图要分别设计量化矩阵。

还需要考虑应用的场景,我们之前接触的Nikon矩阵和Cannon矩阵是用来给照相机使用的,Q矩阵则是计算机压缩图象
成jpeg格式时使用的,自然会不相同。

此外还需要考虑压缩比,针对不同的压缩比,设计的量化矩阵不同,如果要求的压缩比小一些,可以多保留一些分量来
保障图片质量。

关于量化矩阵的设计,我查到一种基于Q矩阵的改进策略。设原本的量化矩阵为Qbase(即我们之前使用的Q),引入质量参数quality,0表示质量最差,100表示质量最好。

计算新矩阵$Q_{new}$的方式如下:

1.定义S,if (Q < 50), then S = 5000/Q, else S = 200 – 2*Q.

2.$Q_{new}$的规模与Qbase相同,$Q_{new}$在(i,j)处的元素可以确定为
\begin{equation}
Q_{new}[i,j] = floor((S * Qbase[i,j] + 50) / 100)
\end{equation}

如果Q = 50,则我们得到的$Q_{new}$与原来的Q完全相同,因此,我们可以设置不同的Q来获取不同质量的图片,当然
当Q取大一些的时候,恢复图片的PSNR更高,但是压缩比更低,如果Q小一些,PSNR会变小,但是压缩比会增大。

还有一种基于机器学习的方法可以用于量化矩阵的求取,可以对于需要压缩的某一类图象,训练Q矩阵,根据相同压缩比
下的PSNR来作为神经网络的loss,最终可以获得专门适配于训练集的Q矩阵。但是用这个方法得到Q矩阵消耗时间比较长,
而且Q矩阵的通用性不高,往往只能用到某一类图象,而且神经网络容易得到极大值而不是最大值,这也是一个隐患。
\section{Nikon矩阵和Cannon矩阵的应用}
将Nikon矩阵和Cannon矩阵应用在lena.bmp上,可以得到两者的全局PSNR和平均PSNR为
\begin{equation}
PSNR_{Cannon} = 49.6190
\end{equation}

\begin{equation}
PSNR_{Cannonaverage} = 87.7596
\end{equation}

\begin{equation}
PSNR_{Nikon} = 51.1356
\end{equation}

\begin{equation}
PSNR_{Nikonaverage} = 88.8013
\end{equation}

\begin{figure}
\centering
\includegraphics[width=0.5\textwidth]{Cannonrecover.png}
\caption{经Cannon矩阵量化后DCT恢复所得图.}
\end{figure}

\begin{figure}
\centering
\includegraphics[width=0.5\textwidth]{Nikonrecover.png}
\caption{经Nikon矩阵量化后DCT恢复所得图.}
\end{figure}


从lena.bmp的实验结果可以看出,Nikon矩阵的信噪比高于Cannon矩阵和Q矩阵。

另外,我选取了几个比较有特点的图片进行信噪比的测量,以下所说的信噪比都是指全局的PSNR度量,而不是针对块平均PSNR度量。选取的图片中,图片A和图片B都是清晰图像,图片C是景深图像,有一部分清晰,有一部分模糊,图片D则是模糊图像,各图在Nikon矩阵和Cannon矩阵量化后的PSNR度量值如表1所示。

\begin{table}
\begin{tabular}{|r|r|r|r|r|}
\hline
\makecell{度量矩阵} & \makecell{A} & \makecell{B} & \makecell{C} & \makecell{D}\\ \hline

\makecell{} & \includegraphics[width=0.22\textwidth]{test1.png} & \includegraphics[width=0.22\textwidth]{test2.png} & \includegraphics[width=0.22\textwidth]{test3.png} & \includegraphics[width=0.22\textwidth]{test4.png}\\ \hline
Cannon & 51.77 & 53.34 & 53.89 & 59.94 \\ \hline
Nikon & 52.73 & 54.17 & 54.33 & 59.82 \\ \hline
\end{tabular}
\caption{不同矩阵度量下各图的PSNR.}
\end{table}

由表1结果可以看出,Nikon矩阵在清晰图像上的效果强于Cannon矩阵量化后的效果,但对于带有景深的图像来说两者差距变小,对于像D那样的模糊图像,两者几乎没有差距。
\end{document}

