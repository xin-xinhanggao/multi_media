\documentclass{ctexart}
\usepackage{graphicx}
\usepackage{amsmath}
\usepackage{listings}
\usepackage{color}
\usepackage{makecell}

\definecolor{mygreen}{rgb}{0,0.6,0}
\definecolor{mygray}{rgb}{0.5,0.5,0.5}
\definecolor{mymauve}{rgb}{0.58,0,0.82}

\lstset{ %
  backgroundcolor=\color{white},   % choose the background color; you must add \usepackage{color} or \usepackage{xcolor}; should come as last argument
  basicstyle=\footnotesize,        % the size of the fonts that are used for the code
  breakatwhitespace=false,         % sets if automatic breaks should only happen at whitespace
  breaklines=true,                 % sets automatic line breaking
  captionpos=b,                    % sets the caption-position to bottom
  commentstyle=\color{mygreen},    % comment style
  deletekeywords={...},            % if you want to delete keywords from the given language
  escapeinside={\%*}{*)},          % if you want to add LaTeX within your code
  extendedchars=true,              % lets you use non-ASCII characters; for 8-bits encodings only, does not work with UTF-8
  frame=single,	                   % adds a frame around the code
  keepspaces=true,                 % keeps spaces in text, useful for keeping indentation of code (possibly needs columns=flexible)
  keywordstyle=\color{blue},       % keyword style
  language=Octave,                 % the language of the code
  morekeywords={*,...},           % if you want to add more keywords to the set
  numbers=left,                    % where to put the line-numbers; possible values are (none, left, right)
  numbersep=5pt,                   % how far the line-numbers are from the code
  numberstyle=\tiny\color{mygray}, % the style that is used for the line-numbers
  rulecolor=\color{black},         % if not set, the frame-color may be changed on line-breaks within not-black text (e.g. comments (green here))
  showspaces=false,                % show spaces everywhere adding particular underscores; it overrides 'showstringspaces'
  showstringspaces=false,          % underline spaces within strings only
  showtabs=false,                  % show tabs within strings adding particular underscores
  stepnumber=2,                    % the step between two line-numbers. If it's 1, each line will be numbered
  stringstyle=\color{mymauve},     % string literal style
  tabsize=2,	                   % sets default tabsize to 2 spaces
  title=\lstname                   % show the filename of files included with \lstinputlisting; also try caption instead of title
}

\author{2014011355 辛杭高}
\title{图像视频大作业一 Part III}
\begin{document}
\maketitle
本部分实验是针对视频的运动物体检测,我从视频中截取了10帧以图象格式存储,相邻的帧间隔1秒,存储于src/img文件夹中,作为运动物体检测的原材料。

\section{Usage}
本实验由matlab编码完成,在exp3/src下直接运行exp3.m即可,运行的时候必须保证img文件夹在同一目录,
而且result文件夹也应该在这一目录(result可以是空文件夹,用于接收运动检测结果)。

输出结果为9个参数,前4个为使用pixel matching在保留1,$\frac{1}{4}$,$\frac{1}{16}$,
$\frac{1}{64}$比例的参数时进行匹配的平均MSE,后4个为使用DCT matching在保留1,$\frac{1}{4}$,$\frac{1}{16}$,$\frac{1}{64}$比例的参数时进行匹配的平均MSE。如64 dct match :296.4349表示保留$\frac{1}{64}$的参数进行匹配时各图的平均MSE,最后一个参数为Haar小波匹配结果的MSE。

最后还会plot两张图,表示两种方法下各帧MSE随帧数的变化。

\section{Pixel Match}
Pixel match的原理为在给定位置的邻域内,对所有16 * 16的像素块进行遍历,寻找在欧式距离意义下与原运动部分最
接近的像素块。

\begin{table}
\Large 
\begin{tabular}{|r|r|r|}
\hline

1 \includegraphics[width=0.35\textwidth]{pixelmatch1.png} & 2\includegraphics[width=0.35\textwidth]{pixelmatch2.png} & 3\includegraphics[width=0.35\textwidth]{pixelmatch3.png} \\ \hline

4\includegraphics[width=0.35\textwidth]{pixelmatch4.png} & 5\includegraphics[width=0.35\textwidth]{pixelmatch5.png} & 6\includegraphics[width=0.35\textwidth]{pixelmatch6.png} \\ \hline

7\includegraphics[width=0.35\textwidth]{pixelmatch7.png} & 8\includegraphics[width=0.35\textwidth]{pixelmatch8.png} & 9\includegraphics[width=0.35\textwidth]{pixelmatch9.png} \\ \hline

\end{tabular}
\caption{Pixel matching在各帧上的结果(用黑色标出).}
\end{table}
表1中列出了各张图片标出运动物体的结果。

各帧MSE随帧数的变化图如图1所示。
\begin{figure}
\centering
\includegraphics[width=0.5\textwidth]{pixel_plot.png}
\caption{Pixel Match下各帧MSE随帧数变化.}
\end{figure}

\section{Dct Match}
Compression domain block matching的原理是在给定位置的邻域内,对所有16 * 16的像素块进行遍历,对比每个
像素块DCT之后的结果与原运动部分的距离,寻找距离最短的像素块。

表2中列出了各张图片标出运动物体的结果。

\begin{table}
\Large 
\begin{tabular}{|r|r|r|}
\hline

1 \includegraphics[width=0.35\textwidth]{dctmatch1.png} & 2\includegraphics[width=0.35\textwidth]{dctmatch2.png} & 3\includegraphics[width=0.35\textwidth]{dctmatch3.png} \\ \hline

4\includegraphics[width=0.35\textwidth]{dctmatch4.png} & 5\includegraphics[width=0.35\textwidth]{dctmatch5.png} & 6\includegraphics[width=0.35\textwidth]{dctmatch6.png} \\ \hline

7\includegraphics[width=0.35\textwidth]{dctmatch7.png} & 8\includegraphics[width=0.35\textwidth]{dctmatch8.png} & 9\includegraphics[width=0.35\textwidth]{dctmatch9.png} \\ \hline

\end{tabular}
\caption{Dct matching在各帧上的结果(用黑色标出).}
\end{table}

各帧MSE随帧数的变化图如图2所示。
\begin{figure}
\centering
\includegraphics[width=0.5\textwidth]{dct_plot.png}
\caption{Dct Match下各帧MSE随帧数变化.}
\end{figure}

\section{Pixel Match与Dct Match的对比}
从图1和图2可以看出,两者在各个帧下的结果几乎完全一致,MSE图也几乎一致。但理论上来说,在保留所有参数的情况
下pixel match的性能应该能在MSE的考量下达到最优。

虽然像素序列可以和DCT序列相互转化,并且具有唯一性,但是由于数值精度的原因,使用像素序列匹配的结果应该在
MSE指标下优于DCT后进行比较的结果。在本次实验中我选取了小车的中部,可以看出两个匹配都找出了小车的中部,但
这也归功于16 * 16的小块和小车的尺寸比较吻合,所以干扰比较小。

如果我们只选取部分参数进行匹配,对进行匹配的9个帧的MSE的平均值进行比较,可以获得

\begin{tabular}{|r|r|r|}
\hline
\makecell{保留比例} & \makecell{Pixel Match} & \makecell{Dct Match} \\ \hline

\makecell{1} & \makecell{270.54} & \makecell{270.54} \\ \hline

\makecell{$\frac{1}{4}$} & \makecell{293.15} & \makecell{270.54} \\ \hline

\makecell{$\frac{1}{16}$} & \makecell{295.85} & \makecell{276.97} \\ \hline

\makecell{$\frac{1}{64}$} & \makecell{8777.36} & \makecell{302.14} \\ \hline
\end{tabular}

可以看出,当丢弃一些数据进行匹配的时候DCT开始展示出明显的优势,在这种情况下DCT得到的结果将更
符合运动物体运动轨迹,而pixel match已经开始出现明显错误。当然,这一结果也与我们的丢失策略有
关系,在本实验中,我丢失DCT参数时集中丢弃了高频分量,所以对结果的影响不是很大。对于Pixel
匹配,我采用的方式是隔若干行若干列保留一个像素点。

表3和表4列出了仅仅保留$\frac{1}{64}$参数的情况下Pixel匹配和DCT匹配的差异。

\begin{table}
\Large 
\begin{tabular}{|r|r|r|}
\hline

1 \includegraphics[width=0.2\textwidth]{6pixelmatch1.png} & 2\includegraphics[width=0.2\textwidth]{6pixelmatch2.png} & 3\includegraphics[width=0.2\textwidth]{6pixelmatch3.png} \\ \hline

4\includegraphics[width=0.2\textwidth]{6pixelmatch4.png} & 5\includegraphics[width=0.2\textwidth]{6pixelmatch5.png} & 6\includegraphics[width=0.2\textwidth]{6pixelmatch6.png} \\ \hline

7\includegraphics[width=0.2\textwidth]{6pixelmatch7.png} & 8\includegraphics[width=0.2\textwidth]{6pixelmatch8.png} & 9\includegraphics[width=0.2\textwidth]{6pixelmatch9.png} \\ \hline

\end{tabular}
\caption{保留$\frac{1}{64}$参数时Pixel matching结果.}
\end{table}


\begin{table}
\Large 
\begin{tabular}{|r|r|r|}
\hline

1 \includegraphics[width=0.2\textwidth]{6dctmatch1.png} & 2\includegraphics[width=0.2\textwidth]{6dctmatch2.png} & 3\includegraphics[width=0.2\textwidth]{6dctmatch3.png} \\ \hline

4\includegraphics[width=0.2\textwidth]{6dctmatch4.png} & 5\includegraphics[width=0.2\textwidth]{6dctmatch5.png} & 6\includegraphics[width=0.2\textwidth]{6dctmatch6.png} \\ \hline

7\includegraphics[width=0.2\textwidth]{6dctmatch7.png} & 8\includegraphics[width=0.2\textwidth]{6dctmatch8.png} & 9\includegraphics[width=0.2\textwidth]{6dctmatch9.png} \\ \hline

\end{tabular}
\caption{保留$\frac{1}{64}$参数时Dct matching的结果.}
\end{table}

\clearpage
\section{其他匹配算法--Haar小波匹配算法}
在src/haarmatch.m中我实现了haar小波匹配算法,与DCT不同的是,我们不再求16 * 16小块的DCT变换,转
而求每个小块的二维haar小波变换。在matlab中,haar小波变换可以直接调用库函数

\begin{equation}
[a,h,v,d] = haart2(x)
\end{equation}
针对16*16的二维块,其结果中a为一个实数,h包含一个8*8的矩阵,一个4 * 4的矩阵,一个2*2的矩阵和一个实数,v、d与h的规模相同。可以从其中选取某些参数作为我们的比较因子。

在本实验中,我选取了4个实数作为一个16*16二维块的特征,来与之前$\frac{1}{64}$比例的参数进行对比。
对比后的结果显示在表5中。
\begin{table}
\Large 
\begin{tabular}{|r|r|r|}
\hline

1 \includegraphics[width=0.2\textwidth]{haarmatch1.png} & 2\includegraphics[width=0.2\textwidth]{haarmatch2.png} & 3\includegraphics[width=0.2\textwidth]{haarmatch3.png} \\ \hline

4\includegraphics[width=0.2\textwidth]{haarmatch4.png} & 5\includegraphics[width=0.2\textwidth]{haarmatch5.png} & 6\includegraphics[width=0.2\textwidth]{haarmatch6.png} \\ \hline

7\includegraphics[width=0.2\textwidth]{haarmatch7.png} & 8\includegraphics[width=0.2\textwidth]{haarmatch8.png} & 9\includegraphics[width=0.2\textwidth]{haarmatch9.png} \\ \hline

\end{tabular}
\caption{保留$\frac{1}{64}$参数时Haar matching的结果.}
\end{table}
该结果的平均MSE为
\begin{equation}
MSE_{har} = 572.67
\end{equation}
从MSE的度量结果来看haar匹配的方法要略弱于DCT匹配的方法,但是从人眼视觉的角度来说,这两者几乎没有差别,另外在仅仅保留部分参数的情况下,haar方法容易产生方形像素块,我们要检测的目标也正好是16 * 16的像素块,在一定程度
上影响了haar match的结果,如果检测的目标变为其他形状应该会得到更好的结果。另外,haar参数的选择也可以有其他
的策略,可以匹配出更好的结果。
\end{document}

