\documentclass{ctexart}
\usepackage{graphicx}
\usepackage{amsmath}
\usepackage{listings}
\usepackage{color}
\usepackage{makecell}
\usepackage{subfigure}

\definecolor{mygreen}{rgb}{0,0.6,0}
\definecolor{mygray}{rgb}{0.5,0.5,0.5}
\definecolor{mymauve}{rgb}{0.58,0,0.82}

\lstset{ %
  backgroundcolor=\color{white},   % choose the background color; you must add \usepackage{color} or \usepackage{xcolor}; should come as last argument
  basicstyle=\footnotesize,        % the size of the fonts that are used for the code
  breakatwhitespace=false,         % sets if automatic breaks should only happen at whitespace
  breaklines=true,                 % sets automatic line breaking
  captionpos=b,                    % sets the caption-position to bottom
  commentstyle=\color{mygreen},    % comment style
  deletekeywords={...},            % if you want to delete keywords from the given language
  escapeinside={\%*}{*)},          % if you want to add LaTeX within your code
  extendedchars=true,              % lets you use non-ASCII characters; for 8-bits encodings only, does not work with UTF-8
  frame=single,	                   % adds a frame around the code
  keepspaces=true,                 % keeps spaces in text, useful for keeping indentation of code (possibly needs columns=flexible)
  keywordstyle=\color{blue},       % keyword style
  language=Octave,                 % the language of the code
  morekeywords={*,...},           % if you want to add more keywords to the set
  numbers=left,                    % where to put the line-numbers; possible values are (none, left, right)
  numbersep=5pt,                   % how far the line-numbers are from the code
  numberstyle=\tiny\color{mygray}, % the style that is used for the line-numbers
  rulecolor=\color{black},         % if not set, the frame-color may be changed on line-breaks within not-black text (e.g. comments (green here))
  showspaces=false,                % show spaces everywhere adding particular underscores; it overrides 'showstringspaces'
  showstringspaces=false,          % underline spaces within strings only
  showtabs=false,                  % show tabs within strings adding particular underscores
  stepnumber=2,                    % the step between two line-numbers. If it's 1, each line will be numbered
  stringstyle=\color{mymauve},     % string literal style
  tabsize=2,	                   % sets default tabsize to 2 spaces
  title=\lstname                   % show the filename of files included with \lstinputlisting; also try caption instead of title
}

\author{2014011355 辛杭高}
\title{音频转换实验 Part II}
\begin{document}
\maketitle
本部分设计到对5个音频的转化,使用的工具是matlab straight,转化后的结果存储在result文件夹里,
都以new开头命名,比如newAsen6000.wav表示A文件夹下的sen6000.wav转化后的结果。

\section{各个音频转化时的参数设置}

\begin{tabular}{|r|r|r|}
\hline

\makecell{待转化文件名} & \makecell{基频调整} & \makecell{频率调整}\\ \hline

\makecell{sen6000.wav} & \makecell{1.9953} & \makecell{1.2535}\\ \hline

\makecell{sen6015.wav} & \makecell{2.062} & \makecell{1.2535}\\ \hline

\makecell{sen6028.wav} & \makecell{2.062} & \makecell{1.2535}\\ \hline

\makecell{sen6044.wav} & \makecell{2.062} & \makecell{1.2535}\\ \hline

\makecell{sen6147.wav} & \makecell{1.8682} & \makecell{1.2535}\\ \hline
\end{tabular}


从表中可以看出,几乎所有文件都仅仅通过调整基频和频率就可以把男性的声音转化成女性的声音,而且这些参数十分相近,这主要
是因为所有A类文件的音色,音速和音调基本相同,只是内容不一样,B类文件也是如此,仅仅有内容上的差别。


A类文件的基频相对于B类文件来说要低一些,频率相对于B类文件来说也会低一些,但是两者的时长是基本一样的,所以在进行调整
的时候我主要是对基频和频率进行了调整。图1以A/sen6000.wav为例显示A类文件analysis的结果,图2以B/sen6000.wav为例
显示了B类文件analysis的结果。

另一个有趣的问题是,如果把B转化成A,该怎么设置参数,通过实验发现,几乎可以当前操作的逆操作,即降低基频并且降低频谱。
降低的幅度取当前提升幅度的倒数即可。这也说明了我们目前的转化操作是可逆的。
\begin{figure}
\centering
\includegraphics[width=1.0\textwidth]{Asen6000.png}
\caption{A/sen6000.wav analysis结果.}
\end{figure}

\begin{figure}
\centering
\includegraphics[width=1.0\textwidth]{Bsen6000.png}
\caption{B/sen6000.wav analysis结果.}
\end{figure}
\end{document}

