\documentclass{ctexart}
\usepackage{graphicx}
\usepackage{amsmath}
\usepackage{listings}
\usepackage{color}
\usepackage{makecell}
\usepackage{subfigure}

\definecolor{mygreen}{rgb}{0,0.6,0}
\definecolor{mygray}{rgb}{0.5,0.5,0.5}
\definecolor{mymauve}{rgb}{0.58,0,0.82}

\lstset{ %
  backgroundcolor=\color{white},   % choose the background color; you must add \usepackage{color} or \usepackage{xcolor}; should come as last argument
  basicstyle=\footnotesize,        % the size of the fonts that are used for the code
  breakatwhitespace=false,         % sets if automatic breaks should only happen at whitespace
  breaklines=true,                 % sets automatic line breaking
  captionpos=b,                    % sets the caption-position to bottom
  commentstyle=\color{mygreen},    % comment style
  deletekeywords={...},            % if you want to delete keywords from the given language
  escapeinside={\%*}{*)},          % if you want to add LaTeX within your code
  extendedchars=true,              % lets you use non-ASCII characters; for 8-bits encodings only, does not work with UTF-8
  frame=single,	                   % adds a frame around the code
  keepspaces=true,                 % keeps spaces in text, useful for keeping indentation of code (possibly needs columns=flexible)
  keywordstyle=\color{blue},       % keyword style
  language=Octave,                 % the language of the code
  morekeywords={*,...},           % if you want to add more keywords to the set
  numbers=left,                    % where to put the line-numbers; possible values are (none, left, right)
  numbersep=5pt,                   % how far the line-numbers are from the code
  numberstyle=\tiny\color{mygray}, % the style that is used for the line-numbers
  rulecolor=\color{black},         % if not set, the frame-color may be changed on line-breaks within not-black text (e.g. comments (green here))
  showspaces=false,                % show spaces everywhere adding particular underscores; it overrides 'showstringspaces'
  showstringspaces=false,          % underline spaces within strings only
  showtabs=false,                  % show tabs within strings adding particular underscores
  stepnumber=2,                    % the step between two line-numbers. If it's 1, each line will be numbered
  stringstyle=\color{mymauve},     % string literal style
  tabsize=2,	                   % sets default tabsize to 2 spaces
  title=\lstname                   % show the filename of files included with \lstinputlisting; also try caption instead of title
}

\author{2014011355 辛杭高}
\title{音频转换实验 Part I}
\begin{document}
\maketitle
本部分实验的主要目的是将两段音频分别转换成3段不同的音频,并对新生成的音频进行分析。在对音频进行处理
的时候,我采用了控制变量的思路,即每次只改变频率,基频和时长中的一个因素。生成的结果在result文件夹里,
分别对应改变基频,改变频率,改变时长三个因素中的一个。

\begin{figure}  
\begin{minipage}[t]{0.5\linewidth}  
\centering  
\includegraphics[width=2.2in]{guodegang.png}  
\caption{guodegang.wav分析结果}  
\label{fig:side:a}  
\end{minipage}%  
\begin{minipage}[t]{0.5\linewidth}  
\centering  
\includegraphics[width=2.2in]{shantianfang.png}  
\caption{shantianfang.wav分析结果}  
\label{fig:side:b}  
\end{minipage}  
\end{figure} 

我使用的工具为matlab straight,两个音频文件analysis的结果如图1所示。

\section{改变基频}
基频对应于声学特征中的音高特性。将基频提高到原来的2.028倍可以得到相应的音频文件guodegang\_2.028.wav
和shantianfang\_2.028.wav存储于result/改变基频下。

从人耳的角度来说,提高基频后的声音音调更高,郭德纲的声音显得有些娘娘腔,
但是仍然可以分辨出这是男性的声音。单田芳的声音音调也随之提高,但是感觉浊音更重一些。

\section{改变频率}
频率对应于声学特征中的音色特性。此时我们将郭德纲的声音频率提高到原来的1.1251倍,将单田芳的声音频率提高到原来的2.2914倍,相应
的音频文件存储在result/改变频率下。

从人耳的角度来说,郭德纲的声音提高1.1251倍后变得更加尖亮,有些接近于单田芳原来的声音,依然可以辨别出这是男性的声音。单田芳的声音
频率提高到原来的2.2914倍后虽然可以听清楚音频的内容,但是已经分不清这是男性的声音还是女性的声音,有些接近于魔音通话的匿名音。

\section{改变时长}
时长对应于声学特征中的音速特性。将郭德纲音频的时长变为原来的1.5336倍,将单田芳音频的时长变为原来的0.629倍,相应
的音频文件存储在result/改变时长下。

时长的改变首先会改变原音频文件的大小。另一方面,对于时长变大的音频文件,内容依然可以听清,但是声音更加低沉,略微显得有些中气不足。
对于时长变短的音频文件,声音会变得尖锐,与提高频率与相似的特点,从这个角度来说,或许降低时长的同时降低频率能够对原音频起到“保真”
的作用。
\end{document}

